\setcounter{section}{2}
\section{Vantagem Comparativa Revelada}
\setcounter{subsection}{3}

\subsection{Vantagem Comparativa dos países (Especialização do Comércio)}

Observando o Relatório de Exportações de 2016 da Argentina (Tabela \ref{tab:ex3-2006-ARG}), observa-se que o país tinha uma grande relevância do setor de óleos/gorduras e vegetais nas exportações e, juntamente com o segmento de milho, detinham o  maior indicador de vantagem comparativa do país. De maneira geral, há grande relevância de bens primários do setor agrícola nas exportações de 2016 da Argentina.

Quando observa-se o Azerbaijão (Tabela \ref{tab:ex3-2006-AZE}), percebe-se que o país tinha o setor de extração de petróleo cru e gás natural como o de maior RCA, sendo também o 6º colocado quando comparado ao restante do mundo. Esse setor foi responsável por 87,6\% das exportações em 2016 do país, evidenciando sua importância e que, de fato, há uma vantagem comparativa do Azerbaijão nesse setor. 

Já no caso do Brasil (Tabela \ref{tab:ex3-2006-BRA}), observa-se que o país detém uma composição das exportações bem diversificada, com nenhum bem/serviço passando de 10\% de relevância. No quesito competitividade, em soja o país tem uma grande competitividade, estando em terceiro no ranking global. De maneira geral, o país tem grande relevância de bens primários como minérios e commodities agrícolas.

No contexto de Cuba (Tabela \ref{tab:ex3-2006-CUB}), percebe-se que o país é o líder global no indicador de RCA para a produção de açúcar, apresentando uma forte vantagem competitiva. O produto também é o primeiro lugar no ranking de exportações do país, com 33,3\%, seguido por produtos de tabaco, com 21,9\% e bebidas destiladas, com 9,1\%. Além do açúcar, o país apresenta vantagem competitiva no setor de bebidas destiladas, sendo o 5º RCA mundial, e no de produtos de tabaco, sendo o 3º mundial.

Acerca do México (Tabela \ref{tab:ex3-2006-MEX}), observa-se a grande relevância de veículos motores na exportação do país, além de correlatos, que trazem as posições de 2º e 3º ao México na comparação com o resto do mundo. Em termos de RCA, a indústria com maior índice é a Malte e Licores, mas representam pouco das exportações do país. De maneira geral, observa-se a alta representatividade do setor Industrial Automotivo para as exportações do México, seguido de Outros Equipamentos Elétricos e Vegetais Frescos.

Já a respeito da Nova Zelândia (Tabela \ref{tab:ex3-2006-NZL}), o país se destaca pela produção de produtos laticínios, que são responsáveis por 18,3\% da participação nas exportações e correspondem ao primeiro lugar no ranking mundial de RCA. Além disso, o país apresenta outra vantagem comparativa em serviços de educação, que também ocupa o topo do ranking mundial e é responsável por 5,0\% das exportações. Por fim, o setor de carne processada e vinhos também assumem uma boa posição no ranking global (4º e 5º, respectivamente), entretanto a participação da indústria nas exportações é menor (10,4\% e 2,8\%, respectivamente).

Observando a Ruanda (Tabela \ref{tab:ex3-2006-RWA}), nota-se que o setor com maior participação nas exportações é o de outros produtos de mineração e pedreiras, com 18,8\%, sendo o 17º lugar no ranking mundial de RCA. Ademais, o setor de bebidas, responsável por 13,8\% das exportações, apresenta um RCA alto no ranking global, em 7º lugar. Por outro lado, produtos no qual o país apresenta uma maior vantagem comparativa, como produtos refratários de cerâmica e produtos cereais, 6º e 9º lugar nos rankings globais de RCA, respectivamente, apresentam uma participação nas exportações de 0,5\% somados.

Acerca do caso da Singapura (Tabela \ref{tab:ex3-2006-SGP}), as exportações do país têm alta participação do setor de Produtos Refinados de Petróleo, ocupando 12,62\% das exportações, embora que, com o resto do mundo ocupe a posição 20ª em termos de RCA para este setor, já em Tubos e Válvulas Eletrônicas (2ª maior representatividade das Exportações, com 11,63\%) observa-se uma posição de 7º no ranking global para o segmento. O setor de liderança neste quesito no país é o de Manutenção e Serviços de Reparos, mas representa apenas 1,2\% das Exportações.

No caso do Uruguai (Tabela \ref{tab:ex3-2006-URY}), o país é o primeiro lugar no ranking mundial de RCA para o setor de carne processada e o segundo lugar para os setores de arroz, processamento de couro, soja e gado vivo. O país aproveitou bem suas vantagens comparativas, tendo cerca de 36,4\% das suas exportações focadas nessas atividades, concentrados principalmente em carne processada e produção de soja.

No tocante aos Estados Unidos da América (Tabela \ref{tab:ex3-2006-USA}), observa-se que os setores das Exportações diversos, com participações não ultrapassando os 7,5\%. Os segmentos que mais exportam são Aeronaves e Espaçonaves (7,28\%) e Cobranças pelo uso de Patentes (6,33\%), que positivamente são setores competitivos em termos globais de RCA ocupando respectivamente as posições de 3º e 2º. Além disso, o setor de Serviços Educacionais nos EUA ocupa a posição de RCA global de 3º, mas tem não muita relevância nas exportações (1,81\%). Não apenas isto, releva-se a participação da exportação de commodities agrícolas nas exportações do país, sendo Arroz e Soja respectivamente as posições de 10º e 6º do ranking global de RCA para estes setores.

\subsection{Implicações em Política Comercial.}


Em termos de Política Comercial, é interessante observar o posicionamento dos países em termos de seus produtos e setores que lideram as exportações em suas respectivas Balanças Comerciais. Dentre os países analisados, observa-se aqueles que têm grande dependência de produtos e setores específicos, quanto aqueles com abrangência diversificada de exportados. Há aqueles com baixa competitividade (medido por RCA) de seus exportados e também aqueles que lideram as transações em aspecto global.

Primeiramente, observa-se uma dicotomia entre a diversificação e a especialização. Países como Argentina, Brasil e México possuem uma gama diversificada de produtos de exportação, o que pode mitigar riscos associados à dependência de setores específicos. No entanto, a especialização em setores onde possuem vantagens comparativas, como agrícolas para o Brasil e produtos automotivos para o México, pode ser uma estratégia eficaz para impulsionar a competitividade, sob a dinâmica de especialização.

Outro aspecto crítico é a dependência de commodities, evidente em países como Argentina, Brasil, Cuba e Uruguai, onde uma alta participação de bens primários nas exportações os torna vulneráveis a flutuações nos preços globais. Diversificar a economia e investir em manufatura de valor agregado pode ser uma estratégia para mitigar essa vulnerabilidade.

Além disso, alguns países possuem uma vantagem comparativa distinta em setores específicos. Azerbaijão, Nova Zelândia e Uruguai se destacam na extração de petróleo, produção de laticínios e carne processada, respectivamente. Estratégias de políticas que promovam o desenvolvimento e a expansão desses setores podem impulsionar as exportações e a economia como um todo, buscando uma competitividade global.

Observando individualmente, há muitos países com alta competitividade de alguns setores, mas que estes não são majoritários nas Exportações, e o contrário também, setores não muito competitivos no país que estão muito representados.

Detalhando por grupos similares, observa-se que a Argentina, Azerbajão, Brasil, Cuba, Mexico, Nova Zelândia, Uruguai e EUA são países que têm majoritariamente focado em seus setores competitivos, com algumas exceções em diversificação como a dependência do setor automotivo e correlatos para o México, agrícolas para a Argentina e os derivados da cana de açúcar e tabaco para Cuba.

Há também países que têm uma parcela relevante em setores não competitivos, e que, em uma ótica de especialização global (modelo Ricardiano), seria ineficiente. São eles o setor de bebidas no Brasil, o de peixes processados para Cuba, Vegetais frescos e outros equipamentos elétricos para o México, Outros minérios e pedreiras para Ruanda, produtos derivados de petróleo para Singapura.

Por último, temos países que já exportam bens/serviços que são competitivos, mas não são relevantes em termos de volume das exportações. São eles os ingredientes de alimentação de animais e pets para a Argentina, Outros açucarados para Cuba, Produtos cerâmicos e refratários para Ruanda e Serviços de manutenção e Publicação de mídia gravada para  Singapura.

\subsection{Vantagem Comparativa (consistência no tempo)}

No âmbito da Argentina (Tabela \ref{tab:ex3-tempo-ARG}), percebe-se que ao longo do tempo houve uma evolução no indicador RCA de bens e serviços agrícolas como, por exemplo, milho, cereais e produtos de alimentação animal. Por outro lado, setores que têm vantagem comparativa em relação ao mundo, como gordura vegetal e animal e produção de soja ao longo do tempo, alcançaram um pico entre 2004 e 2008 com uma queda logo em seguida.

No contexto do Azerbaijão (Tabela \ref{tab:ex3-tempo-AZE}), nota-se uma evolução considerável ao longo do tempo no setor de extração de petróleo cru e gás natural, saindo de um RCA de 9,0 em 2000 para 19,6 em 2016. Por outro lado, houve uma diminuição do RCA de alguns produtos extrativistas, como frutas frescas e nozes, com uma considerável queda, enquanto outros como produtos de cereais e vegetais frescos tiveram um leve aumento.

No caso do Brasil (Tabela \ref{tab:ex3-tempo-BRA}), observa-se uma evolução de RCA em setores como milho, soja e açúcar, enquanto outros setores como mineração de ferro e cigarros de tabaco sofreram uma redução em seus valores. Essas mudanças são consistentes quando analisamos o crescimento da fronteira agrícola brasileira e aumento da produtividade no campo.

Em relação à Cuba (Tabela \ref{tab:ex3-tempo-CUB}), é evidente um grande crescimento do indicador de RCA no setor de bebidas destiladas, adoçantes e produtos de tabaco. Ao mesmo tempo, percebe-se um decrescimento no RCA de açúcar, folhas de tabaco e cigarros e uma estabilização de peixes processados. 

Já em relação ao México (Tabela \ref{tab:ex3-tempo-MEX}), no geral, não houveram grandes mudanças no RCA, tendo vegetais frescos, bebidas de malte, veículos automotores e acessórios para automóveis tendo um ligeiro acréscimo, enquanto gado vivo e outros equipamentos eletrônicos apresentaram uma pequena redução.

No tocante à Nova Zelândia (Tabela \ref{tab:ex3-tempo-NZL}), houve um grande aumento no setor de adoçantes, vinhos e serviços de educação. Ademais, produtos laticínios tiveram um aumento relevante até 2012, quando iniciou-se um processo de queda até 2016. Por fim, produtos animais e carne processada apresentaram uma redução. É possível pensar que tais resultados são condizentes a uma possível transição de especialização de produtos agrícolas para produtos e serviços mais sofisticados.

No contexto de Ruanda (Tabela \ref{tab:ex3-tempo-RWA}), houve uma queda muito relevante do RCA de bebidas e uma menos sensível no setor de mineração e pedreira. Por outro lado, houve uma evolução muito grande no setor de produtos refratários de cerâmica, produtos cereais e de moinho de grãos.

No âmbito de Singapura (Tabela \ref{tab:ex3-tempo-SGP}), houve um grande aumento do RCA de publicações de mídia gravadas e aumentos menores nos setores de petróleo refinado, químicos e serviços de manutenção. De forma contrária, houve queda no indicador do setor de tubos de válvulas eletrônicas.

No caso do Uruguai (Tabela \ref{tab:ex3-tempo-URY}), percebe-se um grande aumento no RCA de arroz, gado vivo e produção de soja, uma estagnação nos setores de carne processada e uma diminuição do setor de processamento de couro. 

Por fim, nos Estados Unidos (Tabela \ref{tab:ex3-tempo-USA}) houve um crescimento no RCA de aeronaves e espaçonaves e serviços de educação, enquanto praticamente não houve mudanças em ingredientes de alimentação para pets e soja, e uma diminuição de encargos pelo uso de propriedades intelectuais. 

\clearpage
\section{Dotação de Fatores e Vantagem Comparativa}
\setcounter{subsection}{4}

\subsection{}

\begin{figure}[!h]
    \caption{A}
    \includegraphics*[width = 0.8\linewidth]{../plots/rca_hc.png}

\end{figure}


\begin{figure}[!h]
    \caption{A}
    \includegraphics*[width = 0.8\linewidth]{../plots/rca_k_constant.png}

\end{figure}

\begin{figure}[!h]
    \caption{A}
    \includegraphics*[width = 0.8\linewidth]{../plots/rca_k_ppp.png}

\end{figure}

\begin{figure}[!h]
    \caption{A}
    \includegraphics*[width = 0.8\linewidth]{../plots/rca_land.png}

\end{figure}

\subsection{}
