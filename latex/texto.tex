\setcounter{section}{2}
\section{Vantagem Comparativa Revelada}
\setcounter{subsection}{3}

\subsection{Vantagem Comparativa dos países (Especialização do Comércio)}

Observando o Relatório de Exportações de 2016 da Argentina (Tabela \ref{tab:ex3-2006-ARG}), observa-se que o país tinha uma grande relevância do setor de óleos/gorduras e vegetais nas exportações e, juntamente com o segmento de milho, detinham o  maior indicador de vantagem comparativa do país. De maneira geral, há grande relevância de bens primários do setor agrícola nas exportações de 2016 da Argentina.

Quando observa-se o Azerbaijão (Tabela \ref{tab:ex3-2006-AZE}), percebe-se que o país tinha o setor de extração de petróleo cru e gás natural como o de maior RCA, sendo também o 6º colocado quando comparado ao restante do mundo. Esse setor foi responsável por 87,6\% das exportações em 2016 do país, evidenciando sua importância e que, de fato, há uma vantagem comparativa do Azerbaijão nesse setor. 

Já no caso do Brasil (Tabela \ref{tab:ex3-2006-BRA}), observa-se que o país detém uma composição das exportações bem diversificada, com nenhum bem/serviço passando de 10\% de relevância. No quesito competitividade, em soja o país tem uma grande competitividade, estando em terceiro no ranking global. De maneira geral, o país tem grande relevância de bens primários como minérios e commodities agrícolas.

No contexto de Cuba (Tabela \ref{tab:ex3-2006-CUB}), percebe-se que o país é o líder global no indicador de RCA para a produção de açúcar, apresentando uma forte vantagem competitiva. O produto também é o primeiro lugar no ranking de exportações do país, com 33,3\%, seguido por produtos de tabaco, com 21,9\% e bebidas destiladas, com 9,1\%. Além do açúcar, o país apresenta vantagem competitiva no setor de bebidas destiladas, sendo o 5º RCA mundial, e no de produtos de tabaco, sendo o 3º mundial.

Acerca do México (Tabela \ref{tab:ex3-2006-MEX}), observa-se a grande relevância de veículos motores na exportação do país, além de correlatos, que trazem as posições de 2º e 3º ao México na comparação com o resto do mundo. Em termos de RCA, a indústria com maior índice é a Malte e Licores, mas representam pouco das exportações do país. De maneira geral, observa-se a alta representatividade do setor Industrial Automotivo para as exportações do México, seguido de Outros Equipamentos Elétricos e Vegetais Frescos.

Já a respeito da Nova Zelândia (Tabela \ref{tab:ex3-2006-NZL}), o país se destaca pela produção de produtos laticínios, que são responsáveis por 18,3\% da participação nas exportações e correspondem ao primeiro lugar no ranking mundial de RCA. Além disso, o país apresenta outra vantagem comparativa em serviços de educação, que também ocupa o topo do ranking mundial e é responsável por 5,0\% das exportações. Por fim, o setor de carne processada e vinhos também assumem uma boa posição no ranking global (4º e 5º, respectivamente), entretanto a participação da indústria nas exportações é menor (10,4\% e 2,8\%, respectivamente).

Observando a Ruanda (Tabela \ref{tab:ex3-2006-RWA}), nota-se que o setor com maior participação nas exportações é o de outros produtos de mineração e pedreiras, com 18,8\%, sendo o 17º lugar no ranking mundial de RCA. Ademais, o setor de bebidas, responsável por 13,8\% das exportações, apresenta um RCA alto no ranking global, em 7º lugar. Por outro lado, produtos no qual o país apresenta uma maior vantagem comparativa, como produtos refratários de cerâmica e produtos cereais, 6º e 9º lugar nos rankings globais de RCA, respectivamente, apresentam uma participação nas exportações de 0,5\% somados.

Acerca do caso da Singapura (Tabela \ref{tab:ex3-2006-SGP}), as exportações do país têm alta participação do setor de Produtos Refinados de Petróleo, ocupando 12,62\% das exportações, embora que, com o resto do mundo ocupe a posição 20ª em termos de RCA para este setor, já em Tubos e Válvulas Eletrônicas (2ª maior representatividade das Exportações, com 11,63\%) observa-se uma posição de 7º no ranking global para o segmento. O setor de liderança neste quesito no país é o de Manutenção e Serviços de Reparos, mas representa apenas 1,2\% das Exportações.

No caso do Uruguai (Tabela \ref{tab:ex3-2006-URY}), o país é o primeiro lugar no ranking mundial de RCA para o setor de carne processada e o segundo lugar para os setores de arroz, processamento de couro, soja e gado vivo. O país aproveitou bem suas vantagens comparativas, tendo cerca de 36,4\% das suas exportações focadas nessas atividades, concentrados principalmente em carne processada e produção de soja.

No tocante aos Estados Unidos da América (Tabela \ref{tab:ex3-2006-USA}), observa-se que os setores das Exportações diversos, com participações não ultrapassando os 7,5\%. Os segmentos que mais exportam são Aeronaves e Espaçonaves (7,28\%) e Cobranças pelo uso de Patentes (6,33\%), que positivamente são setores competitivos em termos globais de RCA ocupando respectivamente as posições de 3º e 2º. Além disso, o setor de Serviços Educacionais nos EUA ocupa a posição de RCA global de 3º, mas tem não muita relevância nas exportações (1,81\%). Não apenas isto, releva-se a participação da exportação de commodities agrícolas nas exportações do país, sendo Arroz e Soja respectivamente as posições de 10º e 6º do ranking global de RCA para estes setores.

\subsection{Implicações em Política Comercial.}


Em termos de Política Comercial, é interessante observar o posicionamento dos países em termos de seus produtos e setores que lideram as exportações em suas respectivas Balanças Comerciais. Dentre os países analisados, observa-se aqueles que têm grande dependência de produtos e setores específicos, quanto aqueles com abrangência diversificada de exportados. Há aqueles com baixa competitividade (medido por RCA) de seus exportados e também aqueles que lideram as transações em aspecto global.

Primeiramente, observa-se uma dicotomia entre a diversificação e a especialização. Países como Argentina, Brasil e México possuem uma gama diversificada de produtos de exportação, o que pode mitigar riscos associados à dependência de setores específicos. No entanto, a especialização em setores onde possuem vantagens comparativas, como agrícolas para o Brasil e produtos automotivos para o México, pode ser uma estratégia eficaz para impulsionar a competitividade, sob a dinâmica de especialização.

Outro aspecto crítico é a dependência de commodities, evidente em países como Argentina, Brasil, Cuba e Uruguai, onde uma alta participação de bens primários nas exportações os torna vulneráveis a flutuações nos preços globais. Diversificar a economia e investir em manufatura de valor agregado pode ser uma estratégia para mitigar essa vulnerabilidade.

Além disso, alguns países possuem uma vantagem comparativa distinta em setores específicos. Azerbaijão, Nova Zelândia e Uruguai se destacam na extração de petróleo, produção de laticínios e carne processada, respectivamente. Estratégias de políticas que promovam o desenvolvimento e a expansão desses setores podem impulsionar as exportações e a economia como um todo, buscando uma competitividade global.

Observando individualmente, há muitos países com alta competitividade de alguns setores, mas que estes não são majoritários nas Exportações, e o contrário também, setores não muito competitivos no país que estão muito representados.

Detalhando por grupos similares, observa-se que a Argentina, Azerbajão, Brasil, Cuba, Mexico, Nova Zelândia, Uruguai e EUA são países que têm majoritariamente focado em seus setores competitivos, com algumas exceções em diversificação como a dependência do setor automotivo e correlatos para o México, agrícolas para a Argentina e os derivados da cana de açúcar e tabaco para Cuba.

Há também países que têm uma parcela relevante em setores não competitivos, e que, em uma ótica de especialização global (modelo Ricardiano), seria ineficiente. São eles o setor de bebidas no Brasil, o de peixes processados para Cuba, Vegetais frescos e outros equipamentos elétricos para o México, Outros minérios e pedreiras para Ruanda, produtos derivados de petróleo para Singapura.

Por último, temos países que já exportam bens/serviços que são competitivos, mas não são relevantes em termos de volume das exportações. São eles os ingredientes de alimentação de animais e pets para a Argentina, Outros açucarados para Cuba, Produtos cerâmicos e refratários para Ruanda e Serviços de manutenção e Publicação de mídia gravada para  Singapura.

\subsection{Vantagem Comparativa (consistência no tempo)}

No âmbito da Argentina (Tabela \ref{tab:ex3-tempo-ARG}), percebe-se que ao longo do tempo houve uma evolução no indicador RCA de bens e serviços agrícolas como, por exemplo, milho, cereais e produtos de alimentação animal. Por outro lado, setores que têm vantagem comparativa em relação ao mundo, como gordura vegetal e animal e produção de soja ao longo do tempo, alcançaram um pico entre 2004 e 2008 com uma queda logo em seguida.

No contexto do Azerbaijão (Tabela \ref{tab:ex3-tempo-AZE}), nota-se uma evolução considerável ao longo do tempo no setor de extração de petróleo cru e gás natural, saindo de um RCA de 9,0 em 2000 para 19,6 em 2016. Por outro lado, houve uma diminuição do RCA de alguns produtos extrativistas, como frutas frescas e nozes, com uma considerável queda, enquanto outros como produtos de cereais e vegetais frescos tiveram um leve aumento.

No caso do Brasil (Tabela \ref{tab:ex3-tempo-BRA}), observa-se uma evolução de RCA em setores como milho, soja e açúcar, enquanto outros setores como mineração de ferro e cigarros de tabaco sofreram uma redução em seus valores. Essas mudanças são consistentes quando analisamos o crescimento da fronteira agrícola brasileira e aumento da produtividade no campo.

Em relação à Cuba (Tabela \ref{tab:ex3-tempo-CUB}), é evidente um grande crescimento do indicador de RCA no setor de bebidas destiladas, adoçantes e produtos de tabaco. Ao mesmo tempo, percebe-se um decrescimento no RCA de açúcar, folhas de tabaco e cigarros e uma estabilização de peixes processados. 

Já em relação ao México (Tabela \ref{tab:ex3-tempo-MEX}), no geral, não houveram grandes mudanças no RCA, tendo vegetais frescos, bebidas de malte, veículos automotores e acessórios para automóveis tendo um ligeiro acréscimo, enquanto gado vivo e outros equipamentos eletrônicos apresentaram uma pequena redução.

No tocante à Nova Zelândia (Tabela \ref{tab:ex3-tempo-NZL}), houve um grande aumento no setor de adoçantes, vinhos e serviços de educação. Ademais, produtos laticínios tiveram um aumento relevante até 2012, quando iniciou-se um processo de queda até 2016. Por fim, produtos animais e carne processada apresentaram uma redução. É possível pensar que tais resultados são condizentes a uma possível transição de especialização de produtos agrícolas para produtos e serviços mais sofisticados.

No contexto de Ruanda (Tabela \ref{tab:ex3-tempo-RWA}), houve uma queda muito relevante do RCA de bebidas e uma menos sensível no setor de mineração e pedreira. Por outro lado, houve uma evolução muito grande no setor de produtos refratários de cerâmica, produtos cereais e de moinho de grãos.

No âmbito de Singapura (Tabela \ref{tab:ex3-tempo-SGP}), houve um grande aumento do RCA de publicações de mídia gravadas e aumentos menores nos setores de petróleo refinado, químicos e serviços de manutenção. De forma contrária, houve queda no indicador do setor de tubos de válvulas eletrônicas.

No caso do Uruguai (Tabela \ref{tab:ex3-tempo-URY}), percebe-se um grande aumento no RCA de arroz, gado vivo e produção de soja, uma estagnação nos setores de carne processada e uma diminuição do setor de processamento de couro. 

Por fim, nos Estados Unidos (Tabela \ref{tab:ex3-tempo-USA}) houve um crescimento no RCA de aeronaves e espaçonaves e serviços de educação, enquanto praticamente não houve mudanças em ingredientes de alimentação para pets e soja, e uma diminuição de encargos pelo uso de propriedades intelectuais. 

\clearpage
\section{Dotação de Fatores e Vantagem Comparativa}
\setcounter{subsection}{4}

\subsection{Vantagem comparativa revelada}

\begin{figure}[!h]
    \centering
    \caption{Dispersão da vantagem comparativa revelada}
    \includegraphics*[width = 0.8\linewidth]{../plots/rca_hc.png}

\end{figure}


\begin{figure}[!h]
    \centering
    \caption{Dispersão da vantagem comparativa revelada}
    \includegraphics*[width = 0.8\linewidth]{../plots/rca_k_constant.png}

\end{figure}

\begin{figure}[!h]
    \centering
    \caption{Dispersão da vantagem comparativa revelada}
    \includegraphics*[width = 0.8\linewidth]{../plots/rca_k_ppp.png}

\end{figure}

\begin{figure}[!h]
    \centering
    \caption{Dispersão da vantagem comparativa revelada}
    \includegraphics*[width = 0.8\linewidth]{../plots/rca_land.png}

\end{figure}

\subsection{Influência dos fatores de produção e a especialização, comparação do Hecksher-Ohlin}

No modelo de Hecksher-Ohlin, os países terão vantagem comparativa de produção de alguns produtos e alguns setores de acordo com a dotação de fatores existentes, dado que não há possibilidade de realocar esses fatores para outros países. Assim, os países acabarão sob uma ótica de incentivos  de produção, se especializando nos setores que mais são intensivos no(s) fator(es) abundante(s) do país e assim sendo mais competitivos.

Deste modo, observando a dotação do fator Terras Agricultáveis, não se observa de maneira clara o esperado pelo modelo, que seria os países com mais terras agricultáveis e especializarem-se na produção dos produtos e no setor agrícola. A explicação para essa diferença pode se dar por políticas protecionistas e o interesse dos países em terem sua produção nacional ou por perto dos alimentos, o que acarreta em subsídios, incentivos à fazendeiros e produtores, e Políticas Comerciais restritivas aos produtos estrangeiros.

Já o fator indicado pelo Índice de Capital Humano, observa-se o esperado pelo modelo, no caso quanto maior o índice, maior foi o RCA para os setores naturalmente mais intensivos em capital humano, sendo eles Manufatura, Mineração e Energia e o setor de Serviços. Sendo mais expresso no primeiro e no último. Essa maior clareza da especialização, em comparação ao das Terras Agricultáveis, pode se dar pela dificuldade de intervenção e restrição comercial nestes setores, afinal, muitas firmas de fato se alocam pelo mundo por conta do menor custo e disponibilidade de mão de obra.

Considerando o fator Capital Físico, há indícios de assimilação com o modelo de Heckscher-Ohlin, refletindo a maior competitividade nos setores de Mineração e Energia (naturalmente intensivos em capital) e também no de Manufatura (que exige capital físico, mas também trabalho). Assim, indicando que de maneira geral quanto maior a disponibilidade de capital físico (à preços nacionais constantes e também sob os níveis de preço do período), maior a competitividade do país nos setores intensivos deste. Deste modo, novamente decorrendo na tese de especialização em setores os quais são intensivos no fator de produção abundante na economia.


\clearpage
\begin{table}
\centering
\caption{Relatório de exportações (ARG)}
\label{tab:ex3-2006-ARG}
\begin{tabular}{p{6cm}p{2cm}p{4cm}p{3cm}}
\toprule
                            Indústria &   RCA & Participação da Indústria nas Exportações &  Posição no Ranking Global RCA \\
\midrule
                                 Corn & 40.23 &                                     7.05\% &                              2 \\
   Vegetable and animal oils and fats & 39.24 &                                    24.38\% &                              1 \\
                     Other sweeteners & 22.90 &                                     0.26\% &                              6 \\
                        Other cereals & 20.77 &                                     1.20\% &                              4 \\
                             Soybeans & 19.49 &                                     5.94\% &                              4 \\
Animal feed ingredients and pet foods & 19.03 &                                     0.64\% &                              2 \\
\bottomrule
\end{tabular}
\end{table}


\begin{table}
\centering
\caption{Relatório de exportações (AZE)}
\label{tab:ex3-AZE}
\begin{tabular}{p{6cm}p{2cm}p{4cm}p{3cm}}
\toprule
                                         Indústria &   RCA & Participação da Indústria nas Exportações &  Posição no Ranking Global RCA \\
\midrule
        Extraction crude petroleum and natural gas & 19.63 &                                    87.56\% &                              6 \\
                                   Cereal products &  6.47 &                                     0.05\% &                             28 \\
                                              Nuts &  5.54 &                                     0.75\% &                             25 \\
                                  Fresh vegetables &  4.05 &                                     1.00\% &                             27 \\
Electricity production, collection, and distrib... &  2.76 &                                     0.39\% &                             24 \\
                                       Fresh fruit &  2.32 &                                     1.09\% &                             58 \\
\bottomrule
\end{tabular}
\end{table}


\begin{table}
\centering
\caption{Relatório de exportações (BRA)}
\label{tab:ex3-BRA}
\begin{tabular}{p{6cm}p{2cm}p{4cm}p{3cm}}
\toprule
                    Indústria &   RCA & Participação da Indústria nas Exportações &  Posição no Ranking Global RCA \\
\midrule
                     Soybeans & 31.70 &                                     9.66\% &                              3 \\
                        Sugar & 29.63 &                                     4.66\% &                              9 \\
          Mining of iron ores & 21.15 &                                     9.82\% &                              4 \\
               Beverages, nec & 16.55 &                                     2.40\% &                             18 \\
Tobacco leaves and cigarettes & 15.48 &                                     1.00\% &                             11 \\
                         Corn & 10.63 &                                     1.86\% &                              8 \\
\bottomrule
\end{tabular}
\end{table}


\begin{table}
\centering
\caption{Relatório de exportações (CUB)}
\label{tab:ex3-CUB}
\begin{tabular}{p{5.5cm}p{3cm}p{3cm}p{3cm}}
\toprule
                                  Indústria &    RCA & Participação da Indústria nas Exportações &  Posição no Ranking Global RCA \\
\midrule
                                      Sugar & 211.41 &                                    33.29\% &                              1 \\
                           Tobacco products & 117.09 &                                    21.90\% &                              3 \\
                           Other sweeteners & 116.95 &                                     1.32\% &                              1 \\
Distilling rectifying \& blending of spirits &  44.99 &                                     9.06\% &                              5 \\
              Processing/preserving of fish &  10.67 &                                     6.44\% &                             43 \\
              Tobacco leaves and cigarettes &   5.75 &                                     0.37\% &                             20 \\
\bottomrule
\end{tabular}
\end{table}


\begin{table}
\centering
\caption{Relatório de exportações (MEX)}
\label{tab:ex3-MEX}
\begin{tabular}{p{5.5cm}p{3cm}p{3cm}p{3cm}}
\toprule
                        Indústria &   RCA & Participação da Indústria nas Exportações &  Posição no Ranking Global RCA \\
\midrule
            Malt liquors and malt & 10.07 &                                     0.98\% &                              9 \\
                 Fresh vegetables &  6.74 &                                     1.67\% &                             17 \\
Other electrical equipment n.e.c. &  4.05 &                                     4.23\% &                             14 \\
Parts/accessories for automobiles &  3.49 &                                     7.94\% &                              2 \\
                      Live Cattle &  3.48 &                                     0.15\% &                             25 \\
                   Motor vehicles &  3.38 &                                    17.38\% &                              3 \\
\bottomrule
\end{tabular}
\end{table}


\begin{table}
\centering
\caption{Relatório de exportações (NZL)}
\label{tab:ex3-NZL}
\begin{tabular}{p{5.5cm}p{3cm}p{3cm}p{3cm}}
\toprule
                                        Indústria &   RCA & Participação da Indústria nas Exportações &  Posição no Ranking Global RCA \\
\midrule
                                   Dairy products & 48.62 &                                    18.63\% &                              1 \\
                                 Other sweeteners & 41.91 &                                     0.47\% &                              2 \\
                               Education services & 18.59 &                                     4.98\% &                              1 \\
                                            Wines & 14.67 &                                     2.82\% &                              5 \\
                    Processing/preserving of meat & 13.79 &                                    10.39\% &                              4 \\
Other meats, livestock products, and live animals & 10.05 &                                     0.65\% &                             13 \\
\bottomrule
\end{tabular}
\end{table}


\begin{table}
\centering
\caption{Relatório de exportações (RWA)}
\label{tab:ex3-RWA}
\begin{tabular}{p{5.5cm}p{3cm}p{3cm}p{3cm}}
\toprule
                  Indústria &   RCA & Participação da Indústria nas Exportações &  Posição no Ranking Global RCA \\
\midrule
             Beverages, nec & 95.03 &                                    13.80\% &                              7 \\
        Grain mill products & 32.85 &                                     6.42\% &                              6 \\
            Cereal products & 29.86 &                                     0.22\% &                              9 \\
  Other mining and quarring & 20.72 &                                    18.76\% &                             17 \\
    Cement lime and plaster & 11.73 &                                     0.75\% &                             11 \\
Refractory ceramic products &  7.63 &                                     0.31\% &                              2 \\
\bottomrule
\end{tabular}
\end{table}


\begin{table}
\centering
\caption{Relatório de exportações (SGP)}
\label{tab:ex3-SGP}
\begin{tabular}{p{6cm}p{2cm}p{4cm}p{3cm}}
\toprule
                             Indústria &   RCA & Participação da Indústria nas Exportações &  Posição no Ranking Global RCA \\
\midrule
          Publishing of recorded media & 22.38 &                                     0.01\% &                              3 \\
            Refined petroleum products &  4.50 &                                    12.62\% &                             20 \\
Maintenance and repair services n.i.e. &  3.32 &                                     1.20\% &                              1 \\
                      Other publishing &  3.31 &                                     0.20\% &                             14 \\
          Electronic valves tubes etc. &  2.70 &                                    11.63\% &                              7 \\
    Basic chemicals except fertilizers &  2.61 &                                     6.04\% &                             14 \\
\bottomrule
\end{tabular}
\end{table}


\begin{table}
\centering
\caption{Relatório de exportações (URY)}
\label{tab:ex3-URY}
\begin{tabular}{p{5.5cm}p{3cm}p{3cm}p{3cm}}
\toprule
                      Indústria &   RCA & Participação da Indústria nas Exportações &  Posição no Ranking Global RCA \\
\midrule
                     Rice (raw) & 48.94 &                                     0.27\% &                              2 \\
                    Live Cattle & 42.65 &                                     1.78\% &                              2 \\
                       Soybeans & 39.63 &                                    12.07\% &                              2 \\
Tanning and dressing of leather & 25.90 &                                     2.84\% &                              2 \\
  Processing/preserving of meat & 25.76 &                                    19.41\% &                              1 \\
            Grain mill products & 23.04 &                                     4.51\% &                              7 \\
\bottomrule
\end{tabular}
\end{table}


\begin{table}
\centering
\caption{Relatório de exportações (USA)}
\label{tab:ex3-USA}
\begin{tabular}{p{5.5cm}p{3cm}p{3cm}p{3cm}}
\toprule
                                         Indústria &  RCA & Participação da Indústria nas Exportações &  Posição no Ranking Global RCA \\
\midrule
                                Education services & 6.77 &                                     1.81\% &                              3 \\
                                        Rice (raw) & 4.57 &                                     0.03\% &                             10 \\
                                          Soybeans & 3.90 &                                     1.19\% &                              6 \\
                           Aircraft and spacecraft & 3.88 &                                     7.28\% &                              3 \\
Charges for the use of intellectual property n.... & 3.75 &                                     6.33\% &                              2 \\
             Animal feed ingredients and pet foods & 3.34 &                                     0.11\% &                             10 \\
\bottomrule
\end{tabular}
\end{table}


\begin{table}
\centering
\caption{Valor do RCA ao longo dos anos para cada indústria (ARG)}
\label{tab:ex3-tempo-ARG}
\begin{tabular}{p{6cm}p{1.5cm}p{1.5cm}p{1.5cm}p{1.5cm}p{1.5cm}}
\toprule
                            Indústria &  2000 &  2004 &  2008 &  2012 &  2016 \\
\midrule
Animal feed ingredients and pet foods &  8.69 &  9.70 & 15.70 & 15.14 & 19.03 \\
                                 Corn & 29.11 & 29.40 & 32.92 & 31.98 & 40.23 \\
                        Other cereals &  5.53 &  4.05 & 11.97 & 33.73 & 20.77 \\
                     Other sweeteners & 52.97 & 39.62 & 34.12 & 28.89 & 22.90 \\
                             Soybeans & 21.93 & 34.12 & 38.54 & 16.67 & 19.49 \\
   Vegetable and animal oils and fats & 38.56 & 46.69 & 38.32 & 29.87 & 39.24 \\
\bottomrule
\end{tabular}
\end{table}


\begin{table}
\centering
\caption{Valor do RCA ao longo dos anos para cada indústria (AZE)}
\begin{tabular}{p{6cm}p{1.5cm}p{1.5cm}p{1.5cm}p{1.5cm}p{1.5cm}}
\toprule
                                         Indústria &  2000 & 2004 & 2008 & 2012 &  2016 \\
\midrule
                                   Cereal products &     - & 4.35 & 5.78 & 1.83 &  6.47 \\
Electricity production, collection, and distrib... &  1.38 & 0.80 & 0.42 & 0.38 &  2.76 \\
        Extraction crude petroleum and natural gas &  9.08 & 9.77 & 9.59 & 8.89 & 19.63 \\
                                       Fresh fruit &  5.23 & 2.43 & 1.30 & 1.81 &  2.32 \\
                                  Fresh vegetables &  1.84 & 1.99 & 1.16 & 1.25 &  4.05 \\
                                              Nuts & 17.04 & 4.46 & 1.54 & 1.68 &  5.54 \\
\bottomrule
\end{tabular}
\end{table}


\begin{table}
\centering
\caption{Valor do RCA ao longo dos anos para cada indústria (BRA)}
\begin{tabular}{p{1cm}p{2cm}p{2cm}p{2cm}p{2cm}p{2cm}p{2cm}}
\toprule
 year &  Beverages, nec &      Corn &  Mining of iron ores &  Soybeans &     Sugar &  Tobacco leaves and cigarettes \\
\midrule
 2000 &       16,397813 &  0,507872 &            36,632391 & 25,300957 & 15,604733 &                      16,789058 \\
 2001 &       15,831111 &  7,121591 &            33,998061 & 27,667071 & 18,723546 &                      15,339127 \\
 2002 &       16,138820 &  5,255953 &            34,477327 & 30,106949 & 22,192472 &                      17,477612 \\
 2003 &       15,829879 &  4,825998 &            31,367386 & 28,948745 & 18,482839 &                      18,740863 \\
 2004 &       17,430383 &  5,530783 &            29,070292 & 32,626766 & 22,226370 &                      19,875761 \\
 2005 &       18,388028 &  1,769669 &            28,029214 & 29,767954 & 23,274529 &                      19,966136 \\
 2006 &       18,317127 &  3,479429 &            30,500093 & 33,682315 & 24,650446 &                      21,451458 \\
 2007 &       18,215263 &  9,252268 &            29,983406 & 29,349512 & 24,922989 &                      20,513144 \\
 2008 &       15,617449 &  4,746000 &            25,892280 & 26,143625 & 21,149351 &                      21,441326 \\
 2009 &       17,473524 &  6,118840 &            25,338591 & 31,883044 & 27,487531 &                      21,831122 \\
 2010 &       17,224435 &  7,503787 &            22,566953 & 22,329141 & 27,965722 &                      17,768380 \\
 2011 &       17,299615 &  6,087232 &            20,736799 & 24,048314 & 27,314409 &                      18,034567 \\
 2012 &       15,860749 & 10,622175 &            21,791503 & 26,382656 & 23,965113 &                      19,479350 \\
 2013 &       14,698351 & 15,248862 &            20,166281 & 31,342506 & 25,420378 &                      17,988787 \\
 2014 &       17,649567 &  8,567944 &            20,262851 & 32,512087 & 24,803427 &                      17,427137 \\
 2015 &       18,042288 & 14,072492 &            21,681891 & 32,613830 & 23,026909 &                      16,635406 \\
 2016 &       16,553806 & 10,630653 &            21,146495 & 31,701207 & 29,627022 &                      15,478390 \\
\bottomrule
\end{tabular}
\end{table}


\begin{table}
\centering
\caption{Valor do RCA ao longo dos anos para cada indústria (CUB)}
\begin{tabular}{p{1cm}p{2cm}p{2cm}p{2cm}p{2cm}p{2cm}p{2cm}}
\toprule
 year &  Distilling rectifying \& blending of spirits &  Other sweeteners &  Processing/preserving of fish &      Sugar &  Tobacco leaves and cigarettes &  Tobacco products \\
\midrule
 2000 &                                     7,737307 &         71,462566 &                      10,424422 & 265,998962 &                      14,435148 &         39,676796 \\
 2001 &                                    10,405575 &         80,551583 &                       9,200777 & 275,437211 &                      16,965304 &         42,810032 \\
 2002 &                                    13,651157 &         53,154607 &                      12,179976 & 305,287070 &                      17,960615 &         44,201895 \\
 2003 &                                    19,506591 &         73,415220 &                      11,001692 & 241,157602 &                      18,776892 &         62,590412 \\
 2004 &                                    27,294252 &        183,031180 &                      13,112083 & 304,384064 &                      17,441029 &         80,627832 \\
 2005 &                                    26,571944 &        112,352522 &                      15,477037 & 137,118574 &                      19,826885 &        106,894354 \\
 2006 &                                    24,482331 &         68,963796 &                      12,357748 & 165,511398 &                      18,205972 &        109,130743 \\
 2007 &                                    28,187712 &         68,613111 &                       9,729649 & 100,533419 &                      10,046884 &         81,927141 \\
 2008 &                                    44,680185 &         78,051648 &                       9,345737 & 154,316528 &                      13,718715 &         81,663426 \\
 2009 &                                    40,109838 &         56,166502 &                       6,756457 & 110,902918 &                      11,231984 &         73,554016 \\
 2010 &                                    39,098128 &         62,004164 &                       6,251020 &  82,696365 &                       5,081839 &         66,699611 \\
 2011 &                                    28,051593 &         54,608300 &                       4,433591 &  75,756868 &                       3,883423 &         53,641864 \\
 2012 &                                    28,820433 &         89,433134 &                       4,699338 & 116,249559 &                       2,553481 &         60,967235 \\
 2013 &                                    33,274426 &         64,020815 &                       3,798340 & 108,989858 &                       4,192022 &         64,731793 \\
 2014 &                                    41,967766 &         95,618783 &                       4,542977 & 157,093095 &                       3,247471 &         74,135433 \\
 2015 &                                    43,688514 &        105,740342 &                       7,352037 & 208,271006 &                       2,731135 &         81,987114 \\
 2016 &                                    44,986653 &        116,954576 &                      10,667473 & 211,407038 &                       5,752775 &        117,085269 \\
\bottomrule
\end{tabular}
\end{table}


\begin{table}
\centering
\caption{Valor do RCA ao longo dos anos para cada indústria (MEX)}
\begin{tabular}{p{1cm}p{2cm}p{2cm}p{2cm}p{2cm}p{2cm}p{2cm}}
\toprule
 year &  Fresh vegetables &  Live Cattle &  Malt liquors and malt &  Motor vehicles &  Other electrical equipment n.e.c. &  Parts/accessories for automobiles \\
\midrule
 2000 &          4,736896 &     4,455616 &               5,956574 &        2,678535 &                           4,242740 &                           1,744448 \\
 2001 &          5,151045 &     5,212065 &               6,550205 &        2,709644 &                           4,295450 &                           1,779541 \\
 2002 &          4,863367 &     3,681671 &               7,232066 &        2,503004 &                           4,784360 &                           1,910746 \\
 2003 &          5,549839 &     6,687325 &               7,795406 &        2,351673 &                           4,653496 &                           1,956078 \\
 2004 &          6,805117 &     8,940070 &               9,213433 &        2,371327 &                           4,707687 &                           2,423034 \\
 2005 &          6,718895 &     7,034946 &               9,988332 &        2,294159 &                           4,688870 &                           2,565857 \\
 2006 &          6,621837 &     5,842475 &              10,782450 &        2,601473 &                           4,607581 &                           2,695746 \\
 2007 &          6,491525 &     5,071302 &              10,014846 &        2,510940 &                           4,954367 &                           2,762096 \\
 2008 &          6,827238 &     3,103271 &               9,034200 &        2,633596 &                           4,193023 &                           2,664227 \\
 2009 &          6,988495 &     4,079612 &               9,482056 &        3,094096 &                           4,013392 &                           2,967194 \\
 2010 &          6,206031 &     3,753508 &               8,443048 &        3,143107 &                           3,601690 &                           2,842092 \\
 2011 &          6,489379 &     4,135322 &               7,788863 &        3,246794 &                           3,651342 &                           2,798693 \\
 2012 &          6,266650 &     4,313430 &               7,426192 &        3,399923 &                           3,899340 &                           2,977367 \\
 2013 &          6,218718 &     2,926969 &               7,572608 &        3,501652 &                           3,917260 &                           3,003720 \\
 2014 &          6,214807 &     3,837057 &               8,872579 &        3,669777 &                           4,099455 &                           3,206390 \\
 2015 &          6,271111 &     4,924988 &               9,514767 &        3,606483 &                           4,193808 &                           3,446563 \\
 2016 &          6,741865 &     3,477651 &              10,074195 &        3,376050 &                           4,049604 &                           3,485633 \\
\bottomrule
\end{tabular}
\end{table}


\begin{table}
\centering
\caption{Valor do RCA ao longo dos anos para cada indústria (NZL)}
\begin{tabular}{p{1cm}p{2cm}p{2cm}p{2cm}p{2cm}p{2cm}p{2cm}}
\toprule
 year &  Dairy products &  Education services &  Other meats, livestock products, and live animals &  Other sweeteners &  Processing/preserving of meat &     Wines \\
\midrule
 2000 &       39,101987 &            0,684455 &                                          18,546149 &          6,317995 &                      19,034917 &  4,218260 \\
 2001 &       43,697598 &            0,721608 &                                          15,456530 &          8,425949 &                      18,719944 &  4,317434 \\
 2002 &       40,440392 &            1,012064 &                                          15,690220 &          7,075001 &                      19,391482 &  4,941572 \\
 2003 &       38,386038 &            1,388398 &                                          16,973309 &          9,142607 &                      20,592928 &  5,497959 \\
 2004 &       41,169103 &            0,903933 &                                          15,182546 &         10,391875 &                      23,611762 &  7,154453 \\
 2005 &       43,948965 &            1,071626 &                                          14,600382 &         16,511472 &                      24,069832 &  9,239094 \\
 2006 &       42,467134 &           24,727646 &                                          13,796375 &         15,490754 &                      19,528855 &  8,774348 \\
 2007 &       43,641964 &           21,843885 &                                          13,047212 &         19,210897 &                      18,013070 & 10,773017 \\
 2008 &       47,390730 &           20,537618 &                                          13,368324 &         19,710995 &                      18,087810 & 11,255403 \\
 2009 &       45,144681 &           18,793838 &                                          11,473922 &         21,410602 &                      16,631020 & 12,360508 \\
 2010 &       47,176673 &           26,482500 &                                          12,025273 &         21,000578 &                      15,657877 & 13,404642 \\
 2011 &       49,242984 &           23,442707 &                                          12,963406 &         22,792548 &                      15,385545 & 12,781506 \\
 2012 &       55,382514 &           24,293870 &                                          12,173354 &         27,987792 &                      14,905054 & 13,544508 \\
 2013 &       53,877679 &           21,266306 &                                          11,384973 &         30,839198 &                      14,552247 & 13,100828 \\
 2014 &       55,198665 &           20,095758 &                                          10,290205 &         29,455565 &                      14,434362 & 13,264015 \\
 2015 &       50,698822 &           19,131580 &                                          10,093026 &         36,630545 &                      16,000185 & 14,278572 \\
 2016 &       48,616058 &           18,591114 &                                          10,045045 &         41,913528 &                      13,788811 & 14,668137 \\
\bottomrule
\end{tabular}
\end{table}


\begin{table}
\centering
\caption{Valor do RCA ao longo dos anos para cada indústria (RWA)}
\begin{tabular}{p{6cm}p{1.5cm}p{1.5cm}p{1.5cm}p{1.5cm}p{1.5cm}}
\toprule
                  Indústria &   2000 &  2004 &   2008 &   2012 &  2016 \\
\midrule
             Beverages, nec & 381.08 & 95.64 & 396.40 & 199.04 & 95.03 \\
    Cement lime and plaster &   0.58 &  1.00 &   5.05 &   4.93 & 11.73 \\
            Cereal products &      - &     - &   5.69 &   1.98 & 29.86 \\
        Grain mill products &   1.58 &  0.15 &   2.57 &  17.04 & 32.85 \\
  Other mining and quarring &  24.84 & 21.28 &  45.07 &  40.55 & 20.72 \\
Refractory ceramic products &      - &     - &   0.01 &   0.11 &  7.63 \\
\bottomrule
\end{tabular}
\end{table}


\begin{table}
\centering
\caption{Valor do RCA ao longo dos anos para cada indústria (SGP)}
\begin{tabular}{p{6cm}p{1.5cm}p{1.5cm}p{1.5cm}p{1.5cm}p{1.5cm}}
\toprule
                         industry\_name & 2000 & 2004 & 2008 & 2012 &  2016 \\
\midrule
    Basic chemicals except fertilizers & 1.12 & 2.73 & 2.08 & 2.45 &  2.61 \\
          Electronic valves tubes etc. & 3.81 & 4.77 & 4.90 & 4.18 &  2.70 \\
Maintenance and repair services n.i.e. &    - &    - & 7.76 & 4.83 &  3.32 \\
                      Other publishing & 1.52 & 2.46 & 1.69 & 2.82 &  3.31 \\
          Publishing of recorded media & 2.19 & 2.48 & 4.10 & 5.95 & 22.38 \\
            Refined petroleum products & 3.77 & 5.25 & 5.32 & 3.99 &  4.50 \\
\bottomrule
\end{tabular}
\end{table}


\begin{table}
\centering
\caption{Valor do RCA ao longo dos anos para cada indústria (URY)}
\begin{tabular}{p{6cm}p{1.5cm}p{1.5cm}p{1.5cm}p{1.5cm}p{1.5cm}}
\toprule
                  industry\_name &  2000 &   2004 &  2008 &  2012 &  2016 \\
\midrule
            Grain mill products & 38.98 &  30.98 & 33.87 & 24.36 & 23.04 \\
                    Live Cattle & 10.72 &  11.77 & 23.85 & 31.28 & 42.65 \\
  Processing/preserving of meat & 24.42 &  31.25 & 29.43 & 23.74 & 25.76 \\
                     Rice (raw) &  0.52 & 116.48 &  4.59 & 68.01 & 48.94 \\
                       Soybeans &  1.22 &  27.21 & 25.78 & 51.75 & 39.63 \\
Tanning and dressing of leather & 38.93 &  47.05 & 31.96 & 24.93 & 25.90 \\
\bottomrule
\end{tabular}
\end{table}


\begin{table}
\centering
\caption{Valor do RCA ao longo dos anos para cada indústria (USA)}
\label{tab:ex3-tempo-USA}
\begin{tabular}{p{6cm}p{1.5cm}p{1.5cm}p{1.5cm}p{1.5cm}p{1.5cm}}
\toprule
                                         Indústria & 2000 & 2004 & 2008 & 2012 & 2016 \\
\midrule
                           Aircraft and spacecraft & 2.97 & 3.89 & 4.23 & 3.76 & 3.88 \\
             Animal feed ingredients and pet foods & 3.40 & 4.35 & 3.94 & 3.71 & 3.34 \\
Charges for the use of intellectual property n.... & 4.47 & 4.82 & 5.01 & 5.12 & 3.75 \\
                                Education services & 5.50 & 4.07 & 4.92 & 6.09 & 6.77 \\
                                        Rice (raw) & 4.70 & 6.06 & 7.26 & 4.62 & 4.57 \\
                                          Soybeans & 3.98 & 4.03 & 4.14 & 4.35 & 3.90 \\
\bottomrule
\end{tabular}
\end{table}
