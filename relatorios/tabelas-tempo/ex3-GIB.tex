\begin{table}
\centering
\caption{Valor do RCA ao longo dos anos para cada indústria (GIB)}
\begin{tabular}{p{6cm}p{1.5cm}p{1.5cm}p{1.5cm}p{1.5cm}p{1.5cm}}
\toprule
                                         Indústria &  2000 &   2004 &   2008 &   2012 &  2016 \\
\midrule
                   Building and repairing of ships & 25.27 &   5.01 &  24.06 &  13.59 & 55.57 \\
       Building/repairing of pleasure/sport. boats & 86.29 & 394.64 & 124.59 & 189.16 & 49.91 \\
                                    Motor vehicles &  0.27 &   3.24 &   1.13 &   0.81 &  2.36 \\
                        Refined petroleum products &  3.26 &   0.72 &   2.07 &  13.22 & 15.23 \\
Telecommunications, computer, and information s... &     - &   0.72 &   1.74 &   2.44 &  3.28 \\
                                 Wooden containers &  1.06 &   1.81 &   3.52 &   2.21 &  3.86 \\
\bottomrule
\end{tabular}
\end{table}
