\begin{table}
\centering
\caption{Valor do RCA ao longo dos anos para cada indústria (FRO)}
\begin{tabular}{p{1cm}p{2cm}p{2cm}p{2cm}p{2cm}p{2cm}p{2cm}}
\toprule
 year &  Building and repairing of ships &  Cordage rope twine and netting &  Food/beverage/tobacco processing machinery &  Other agricultural products, nec &  Prepared animal feeds &  Processing/preserving of fish \\
\midrule
 2000 &                        21,358156 &                       14,588201 &                                    0,220590 &                          7,133044 &              19,051776 &                     116,888472 \\
 2001 &                         3,857796 &                        2,931654 &                                    0,407715 &                          6,571975 &              14,796418 &                     126,604676 \\
 2002 &                         5,300459 &                        3,866160 &                                    2,332313 &                          7,018651 &               5,843323 &                     125,271756 \\
 2003 &                         7,620871 &                        6,882608 &                                    1,621834 &                         11,210187 &              11,869352 &                     129,099268 \\
 2004 &                        21,803530 &                       12,531990 &                                    0,486325 &                         10,582848 &              22,527747 &                     160,870088 \\
 2005 &                        11,008976 &                        6,004869 &                                    0,514402 &                          8,176441 &              37,562922 &                     168,133727 \\
 2006 &                         4,042974 &                       12,052352 &                                    0,665830 &                         11,162521 &              49,368273 &                     158,131406 \\
 2007 &                         5,857925 &                       10,650853 &                                    1,790442 &                         11,552271 &              44,921621 &                     175,680063 \\
 2008 &                        60,161073 &                       20,339470 &                                    0,668304 &                          9,532512 &               5,144937 &                     136,578988 \\
 2009 &                        16,622119 &                       10,940558 &                                    0,502099 &                          7,377615 &              20,277401 &                     156,419629 \\
 2010 &                         4,408612 &                       10,143259 &                                    0,530572 &                          7,532739 &                      - &                     176,913994 \\
 2011 &                         1,470463 &                        8,693735 &                                    0,794415 &                          2,860317 &               2,067861 &                     168,715590 \\
 2012 &                         0,925639 &                       11,135316 &                                    0,554899 &                          4,662803 &               9,683687 &                     177,813170 \\
 2013 &                         5,131045 &                        7,559466 &                                    0,754223 &                          9,255059 &               7,650103 &                     169,138322 \\
 2014 &                         5,904073 &                        6,672253 &                                    0,406308 &                          9,699088 &              10,099783 &                     164,243525 \\
 2015 &                         3,284369 &                       12,708098 &                                    0,291075 &                          8,319100 &               3,304212 &                     162,009426 \\
 2016 &                        15,773670 &                       10,649031 &                                    1,076787 &                         16,180031 &               3,299553 &                     139,963106 \\
\bottomrule
\end{tabular}
\end{table}
