\begin{table}
\centering
\caption{Valor do RCA ao longo dos anos para cada indústria (SLE)}
\label{tab:ex3-tempo-SLE}
\begin{tabular}{p{6cm}p{1.5cm}p{1.5cm}p{1.5cm}p{1.5cm}p{1.5cm}}
\toprule
                    Indústria &  2000 &   2004 &   2008 &   2012 &   2016 \\
\midrule
     Cocoa and cocoa products & 30.64 & 123.95 & 196.05 & 121.16 &  69.72 \\
             Fresh vegetables &     - &   0.07 &   0.04 &   0.00 &   3.34 \\
          Mining of iron ores &     - &   0.00 &      - &  63.72 &  42.14 \\
    Other mining and quarring & 10.64 &  75.92 &  55.48 &  36.08 &  35.92 \\
Processing/preserving of fish &  0.53 &   3.71 &   0.69 &   0.33 &  38.25 \\
 Starches and starch products &  0.14 &   0.03 &   0.00 &   0.04 & 125.60 \\
\bottomrule
\end{tabular}
\end{table}
